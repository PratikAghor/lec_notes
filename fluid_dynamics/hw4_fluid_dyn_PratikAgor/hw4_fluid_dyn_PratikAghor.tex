\documentclass{article}
\usepackage[utf8]{inputenc}
\usepackage{amsmath}
\usepackage{float}
\usepackage{listings}
\usepackage[pdftex]{graphicx}
\usepackage{amssymb}
\usepackage{subcaption}
\usepackage{cancel}
\usepackage{float}
\usepackage{hyperref}
\DeclareMathOperator{\sech}{sech}

\newcommand{\pd}[2]{\frac{\partial{#1}}{\partial{#2}}}
%
\newcommand{\pdd}[2]{\frac{\partial^2{#1}}{\partial{#2}^2}}
%
\newcommand{\pddmixed}[3]{\frac{\partial^2{#1}}{\partial{#2}\partial{#3}}}

%---------------------------------------------------------
\author{Pratik Aghor}
\title{HW $\# 4$: Boundary Layer Analysis of Large $Re$ Flows}
\date{\today}  % Toggle commenting to test

\begin{document}

\maketitle
%----------------------------------------------
\section{$2d$ Laminar Jet:}
%----------------------------------------------
Governing equations are the $2d$ Navier-Stokes equations plus the continuity equation. 
%
\begin{align}\label{eq:NS_dim}
 \cancelto{0 \because \textrm{steady}}{\frac{\partial \underline{u}^{*} }{\partial t^{*}}} + [\underline{u}^{*} . \nabla^{*}] \underline{u}^{*} &= -\frac{1}{\rho^{*}}\cancelto{\frac{\partial p}{\partial y} \hat{e}_{y} \because \textrm{no imposed pressure gradient in x}}{\nabla^{*} p^{*}} + \nu^{*} {\nabla^{*}}^{2}\underline{u}^{*},\\
 %
 \nabla^{*} \cdot \underline{u}^{*} &= 0,
\end{align}
%
where stars denote dimensional quantities. We define the dimensionless quantities as follows:
$x = x^{*}/L, Y = y^{*}/\delta, u = u^{*}/U_{0}, V = v^{*}/V_{0}$, where $Y, V$ are the boundary layer variables. By continuity, $U_{0}/L \sim V_{0}/\delta$, or $\boxed{V_{0} = \epsilon U_{0}}$, with $\epsilon = \delta/L$. We now write the dimensional momentum equations inside the boundary layer (BL) and the relative sizes of terms.
%----------------------------------------------
\subsection{Laminar Jet - Scaling}
%----------------------------------------------

\begin{align}\label{eq:jet-x-mom-scaling}
 \begin{split}
  & u^{*}\frac{\partial u^{*}}{\partial x^{*}} + v^{*} \frac{\partial u^{*}}{\partial y^{*}} = \nu^{*}  \left[ \frac{\partial^{2} u}{\partial x^{2}} + \frac{\partial^{2} u}{\partial Y^{2}}\right]\\
  %
  & \frac{U_{0}^{2}}{L} \quad \quad \quad \frac{U_{0}^{2}}{L} \quad \quad \quad  \frac{\nu ^{*}U_{0}}{L^{2}} \quad \quad  \frac{\nu ^{*}U_{0}}{\epsilon^{2} L^{2}}\\
  %
  & 1 \quad \quad \quad \quad 1 \quad \quad \quad \quad \frac{1}{Re} \quad \quad \quad \frac{1}{\epsilon^{2} Re}
 \end{split}
\end{align}
%
Since physically, we know there is diffusion of vorticity in the $y$ direction and we can see that $ \frac{\partial^{2} u}{\partial Y^{2}} \gg \frac{\partial^{2} u}{\partial x^{2}} $, for keeping the diffusion term at the leading order, we must have $\epsilon \sim O( \frac{1}{\sqrt{Re}} )$. We choose $\boxed{\epsilon = \frac{1}{\sqrt{Re} } }$. Hence, the dimensionless $x$-momentum equation at the leading order takes the following form:
\begin{equation}\label{eq:jet-x-mom-dimless}
 u \frac{\partial u}{\partial x} + V \frac{\partial u}{\partial Y} = \frac{\partial^{2} u}{\partial Y^{2}}. 
\end{equation}
%
Let us do a similar scaling analysis for the $y$-moementum equation:

\begin{align}\label{eq:jet-y-mom-scaling}
 \begin{split}
  & u^{*}\frac{\partial v^{*}}{\partial x^{*}} + v^{*} \frac{\partial v^{*}}{\partial y^{*}} = -\frac{1}{\rho^{*}}\frac{\partial p^{*} }{\partial Y^{*}} + \nu^{*}  \left[ \frac{\partial^{2} v}{\partial x^{2}} + \frac{\partial^{2} v}{\partial Y^{2}}\right]\\
  %
  & \frac{\epsilon U_{0}^{2}}{L} \quad \quad \quad \frac{\epsilon U_{0}^{2}}{L} \quad \quad \quad  \frac{1}{\rho^{*}}\frac{P}{\epsilon L}\quad \quad \quad \frac{\epsilon \nu ^{*}U_{0}}{L^{2}} \quad \quad  \frac{\cancel{\epsilon} \nu ^{*}U_{0}}{\epsilon^{\cancel{2} } L^{2}}\\
  %
  & 1 \quad \quad \quad \quad 1 \quad \quad \quad \frac{P^{*}}{\epsilon^{2}\rho^{*} U_{0}^{2}} \quad \quad \quad \quad \quad \frac{1}{Re} \quad \quad \quad \frac{1}{\epsilon^{2} Re}\\
  %
  & 1 \quad \quad \quad \quad 1 \quad \quad \quad \frac{1}{\epsilon^{2}} \quad \quad \quad \quad \quad \quad \quad \frac{1}{Re} \quad \quad \quad 1\\
 \end{split}
\end{align}
choosing $P^{*} = \rho U_{0}^{2}$, we see that at the leading order, the dimensionless $y$-momentum equation reads:
\begin{equation}\label{eq:jet-y-mom-dimless}
 \frac{\partial p}{\partial y} = 0.
\end{equation}
%
This says that the outer pressure is simply impressed inside the BL, which is often the case in BL analyses. Hence, writing the dimensional governing equations:
\begin{align}\label{eq:jet-gov-eqns-dimless}
 \begin{split}
   u \frac{\partial u}{\partial x} + V \frac{\partial u}{\partial Y} &= \frac{\partial^{2} u}{\partial Y^{2}}\\
   %
    \frac{\partial u}{\partial x} + \frac{\partial V}{\partial Y} &= 0.
 \end{split}
\end{align}
%
%----------------------------------------------
\subsection{Laminar Jet - Conservation of momentum flux}
%----------------------------------------------

Now, the LHS of the $x$-momentum equation (Eqn. \ref{eq:jet-x-mom-dimless}) can be modified as follows:
\begin{equation}\label{eq:jet-modified-x-mom-dimless}
 \frac{\partial (u^{2})}{\partial x} - u \cancel{\frac{\partial u}{\partial x}} + \frac{\partial Vu}{\partial Y} - u \cancel {\frac{\partial V}{\partial Y}} = \frac{\partial^{2} u}{\partial Y^{2}}
\end{equation}
%
In modifying the LHS, we have used the incompressibility of the jet. Integrating Eqn.(\ref{eq:jet-modified-x-mom-dimless}) and using the boundary and symmetry conditions, we obtain:

\begin{align}\label{eq:jet-mom-flux-cons}
 \begin{split}
  & \int_{-\infty}^{\infty}\left[ \frac{\partial (u^{2})}{\partial x} dY + \frac{\partial Vu}{\partial Y} = \frac{\partial^{2} u}{\partial Y^{2}} \right] dY, \\
  %
  & \frac{\partial }{\partial x}\left[ \int_{-\infty}^{\infty} u^{2} dY \right] + \cancelto{0 }{\left[ Vu\right]}_{-\infty}^{\infty} = \cancelto{0 \because \textrm{ symmetry wrt } Y = 0}{\frac{\partial u}{\partial Y}\bigg|_{-\infty}^{\infty}},\\
  %
  & \boxed{\int_{-\infty}^{\infty} u^{2} dY = M},
 \end{split}
\end{align}
%
where $M$ does not vary along the streamwise direction $x$. 
%----------------------------------------------
\subsection{Laminar Jet - Similarity Solution:}
%----------------------------------------------
Since there are no imposed length or time-scales, there is a possibility that a similarity solution might exist. First, we introduce a stream-function
$\psi$, such that $u = \psi_{Y}, v = -\psi_{x}$. The incompressibility is automatically satisfied and the $x$-momentum equation becomes:
\begin{equation}\label{eq:jet-streamfn-x-mom-dimless}
 \psi_{Y} \psi_{xY} - \psi_{x}\psi_{YY} = \psi_{YYY}.
\end{equation} 
%
We now introduce the similarity ansatz:
\begin{equation}\label{eq:jet-similarity-ansatz}
 \psi (x, Y) = F(x) f(\eta), 
\end{equation}
with $\eta = Y/g(x)$. Substituting Eqn.(\ref{eq:jet-similarity-ansatz}) into Eqn.(\ref{eq:jet-mom-flux-cons}), we get a relationship between $F(x)$ and $g(x)$.
\begin{align}
 \begin{split}
 & \int_{-\infty}^{\infty}(\psi_{Y})^{2} dY = M \\
 & \int_{-\infty}^{\infty} F^{2} f'^{2} \cdot \left(\frac{\partial \eta}{\partial Y} \right)^{2} \cdot g d\eta = M\\
 & \frac{F(x)^{2}}{g(x)} \int_{-\infty}^{\infty} f'^{2}d\eta = M,
 \end{split}
\end{align}
here, primes denote differentiation wrt $\eta$. 
As suggested in the problem, setting $\int_{-\infty}^{\infty} f'^{2}d\eta = 2/3$ gives
$\boxed{F(x) = \left(\frac{3M}{2}\right)^{1/2} [g(x)]^{1/2}}$.

Now, we want to substitute Eqn.(\ref{eq:jet-similarity-ansatz}) into Eqn.(\ref{eq:jet-streamfn-x-mom-dimless}), but before doing so, we evaluate the specific derivatives. Note, we treat $x$ and $\eta$ as independent variables. Dots represent derivatives wrt $x$ and primes represent differentiation wrt $\eta$. 
\begin{align}
 \begin{split}
  & \psi_{X} = \left(\frac{3M}{2}\right)^{1/2} \cdot \frac{1}{2} g^{-1/2} \dot{g} \cdot f\\
  %
  & \psi_{Y} = \frac{F}{g} f' =  \left(\frac{3M}{2}\right)^{1/2} \frac{f'}{g^{1/2}},\\
  %
  & \psi_{YY} = \partial_{Y}\left(\left(\frac{3M}{2}\right)^{1/2} \frac{f'}{g^{1/2}} \right) =\left(\frac{3M}{2}\right)^{1/2} \frac{f''}{g^{3/2}},\\
  %
  & \psi_{YYY} = \partial_{Y}\left(\left(\frac{3M}{2}\right)^{1/2} \frac{f''}{g^{3/2}} \right) = \left(\frac{3M}{2}\right)^{1/2} \frac{f'''}{g^{5/2}}.
 \end{split}
\end{align}
Substituting these in Eqn.(\ref{eq:jet-streamfn-x-mom-dimless}) and canceling common factors, we obtain:
\begin{equation}
 -\frac{1}{2} \left(\frac{3M}{2}\right)^{1/2} f'^{2} \dot{g} - \frac{1}{2}\left(\frac{3M}{2}\right)^{1/2} \dot{g} f f' = f''' g^{-1/2}.
\end{equation}
%
To make the above equation independent of $x$, we must have $\dot{g} \sim g^{-1/2}$, which can be easily integrated to yield $\boxed{g(x) \sim x^{2/3} }$. As suggested in the problem, choosing $\boxed{ g(x) = \left(\frac{3M}{2}\right)^{-1/3}(3x)^{2/3} }$, the above equation reduces to:
\begin{equation}\label{eq:f-eqn}
 f'^{2} + f f'' + f''' = 0.
\end{equation}
%
The BC $u =0$ as $y \rightarrow \pm \infty$ reduces to $\boxed{f'(\infty) = 0}$. The symmetry condition at $Y=0$, $\frac{du}{dY} = 0$ reduces to $\boxed{f''(0) = 0}$. Also, by symmetry, we set the $\psi = 0$ streamline at $Y=0$, giving us another required BC $\boxed{f(0) = 0}$. 

Combining $ff'' + f'^{2} = (ff')'$, we get\
\begin{equation}
 f''' + (ff')' = 0
\end{equation}
Integrating once wrt $\eta$, obtain $f'' + ff' = c_{1}$. Since $f''(0) = f(0) = 0$, $\boxed{c_{1} = 0}$. Rewriting $f'' + ff' = 0$ as $f'' + \left(\frac{f^{2}}{2}\right)' = 0$ and integrating once more in $\eta$, 
\begin{equation}
 f'+ \frac{f^{2}}{2} = c_{2}.
\end{equation}
Solving this
\begin{equation}
 f(\eta) = 2A \tanh{A(\eta + k)}
\end{equation}
Since $f(0) = 0$, $\tanh{A(k)} = 0$, giving $\boxed{k = 0}$ $\Rightarrow \boxed{f(\eta) = 2A \tanh{(A\eta)}}  $.\\
%
We had set $\int_{-\infty}^{\infty} f'^{2}d\eta = 2/3$. This yields,
\begin{align}
 \begin{split}
  & \int_{-\infty}^{\infty} f'^{2}d\eta = 2/3, \\
  %
  & 4 A^{4} \int_{-\infty}^{\infty} \sech^{4}{(A\eta)}d\eta = 2/3,\\
  %
  &\frac{2A^{4}}{A}\int_{-\infty}^{\infty} \sech^{4}{\zeta} d\zeta = 1/3,\\
  %
  & 2 A^{3} \cdot 4/3 = 1/3\\
  & \boxed{A = 1/2}.
 \end{split}
\end{align}
%
Now, we can obtain $u(x, Y)$ as follows:
\begin{align}
 \begin{split}
  & u = \psi_{Y} =  \left(\frac{3M}{2}\right)^{1/2} \frac{f'}{g^{1/2}}, \\
  %
  & \boxed{u =\frac{1}{2}\left(\frac{3M^{2}}{4x}\right)^{1/3}\sech^{2}{(\eta/2)} }.
 \end{split}
\end{align}
%----------------------------------------------
\section{Quasi-Geostrophic (QG) Vorticity Equation, Ekman Boundary Layer (BL) in the $\beta$-Plane:}
%----------------------------------------------
From our discussion of Ekman boundary layers that the leading-order interior geostrophic flow ($u_{0}$ and $v_{0}$) is not constrained at leading order; that is, $u_{0}$ and $v_{0}$ are in geostrophic balance with the leading order interior pressure $\pi_{0}$, but these variables are otherwise unknown. This indeterminacy can be remedied by going to the next order in the expansion for the interior flow and making use of our leading-order Ekman boundary layer analysis.
%----------------------------------------------
\subsection{The $\beta$-Plane:}
%----------------------------------------------
The ``Coriolis parameter'' $f$ is given by (twice) the component of the planetary angular velocity in the local $z$-direction. We measure $\theta$ from the equator, so $\theta=0$ corresponds to the equator while at the North pole, $\theta=\pi/2$. 
\begin{align}\label{eq:beta-plane-approx}
 \begin{split}
  & f = 2\Omega \sin{\theta},\\
  %
  & f = 2\Omega \sin{\theta_{0}} + (\Delta \theta) 2\Omega \cos{\theta_{0}} + \textrm{h.o.t.},\\
  %
  &\textrm{noting } r_{0} \Delta \theta = \tilde{y},\\
  %
  & f \approx 2\Omega \sin{\theta_{0}} + \frac{\tilde{y}}{r_{0}} 2\Omega \cos{\theta_{0}},\\
  %
  & \boxed{ f \approx f_{0} + \beta_{0} \tilde{y}},
 \end{split}
\end{align}
where $f_{0} = 2\Omega \sin{\theta_{0}}, \beta_{0} = \frac{2\Omega \cos{\theta_{0}}}{r_{0}}$ are dimensional parameters and $\tilde{y}$ is the local cross-flow (northward) co-ordinate (we assume that the wind is blowing in the $x$-direction).  
%----------------------------------------------
\subsection{Governing Equations:}
%----------------------------------------------
\begin{align}\label{eq:rotating-NS}
 \begin{split}
  & \pd{u}{t} + u \pd{u}{x} + v \pd{u}{y} + w\pd{u}{z} - fv = -\frac{1}{\rho} \pd{\pi}{x} + \nu\left[\pdd{u}{x} + \pdd{u}{y} + \pdd{u}{z}\right],\\
  %
  & \pd{v}{t} + u \pd{v}{x} + v \pd{v}{y} + w\pd{v}{z} + fu = -\frac{1}{\rho} \pd{\pi}{y} + \nu\left[\pdd{v}{x} + \pdd{v}{y} + \pdd{v}{z}\right],\\
  %
  & \pd{w}{t} + u \pd{w}{x} + v \pd{w}{y} + w\pd{w}{z}  = -\frac{1}{\rho} \pd{\pi}{z} + \nu\left[\pdd{w}{x} + \pdd{w}{y} + \pdd{w}{z}\right],\\
  %
  & \pd{u}{x} + \pd{v}{y} + \pd{w}{z} = 0.
 \end{split}
\end{align}
 First, let us look at the Coriolis terms. For example, take $fu$. 
\begin{equation}
 f u = (f_{0}+\beta_{0}\tilde{y})u = f_{0}U_{0}\left(1 + \frac{\beta_{0} L}{f_{0}} y \right)u 
\end{equation}
where $y$ and $U$ on the RHS are dimensionless terms. Simplifying further and taking only the dimensionless version now, 
\begin{equation}
 \left(1 + \frac{\beta_{0} L}{f_{0}} y \right)u  = \bigg(1 + \underbrace{ \frac{\beta_{0} L^{2}}{U_{0}} }_{\beta} \underbrace{ \frac{U_{0}}{L f_{0}} }_{\epsilon} y \bigg)u = (1 + \epsilon \beta y) u, 
\end{equation}
where $\beta =  \frac{\beta_{0} L^{2}}{U_{0}} $ is a new parameter and $\epsilon =  \frac{U_{0}}{L f_{0}}$ is the familiar Rossby number.\\
%
We scale $x, y \sim L, z \sim D, f\sim f_{0}, t\sim L/U_{0}, u, v \sim U_{0}, w \sim W_{0}$. From continuity, it is immediately clear that $\frac{W_{0}}{D} \sim \frac{U_{0}}{L}$ or $W_{0} \sim \Gamma U_{0}$, where $\Gamma = \frac{D}{L}$ is the aspect ratio. \\
%
Let $p \sim P$. For geostrophic balance at the leading order, $\frac{1}{\rho} \frac{\partial \pi}{\partial x} \sim f v$, yielding $\boxed{P = \rho f_{0} U_{0} L}$.
%
Using these scalings, let us non-dimensionalize the $x$-momentum equation:
\begin{align}
 \begin{split}
  & \pd{u}{t} + u \pd{u}{x} + v \pd{u}{y} + w\pd{u}{z} - fv = -\frac{1}{\rho} \pd{\pi}{x} + \nu\left[\pdd{u}{x} + \pdd{u}{y} + \pdd{u}{z}\right],\\
  %
  &\frac{U_{0}^{2}}{L} \quad \quad \frac{U_{0}^{2}}{L} \quad \quad  \frac{U_{0}^{2}}{L} \quad \quad \frac{U_{0}^{2}}{L} \quad f_{0}U_{0} \quad f_{0}U_{0} \quad \frac{\nu \Gamma^{2}U_{0}}{D^{2}} \quad \frac{\nu \Gamma^{2}U_{0}}{D^{2}} \quad \frac{\nu U_{0}}{D^{2}} \\
  & \epsilon \qquad \qquad \epsilon \qquad\epsilon \qquad \qquad \epsilon \qquad 1 \qquad 1 \qquad \qquad \Gamma^{2}E \qquad  \Gamma^{2}E \qquad  E 
 \end{split}
\end{align}
Here, we divided throughout by $f_{0}U_{0}$ to get the coefficients. In the above $\epsilon = \frac{U_{0}}{Lf_{0}}$ is the Rossby number, while $E = \frac{\nu}{f_{0}D_{0}^{2}}$ is the Ekman number. We can do similar analyses for other momenum equations. The governing dimensionless equations now become:
\begin{align}\label{eq:rotating-NS-dimless}
 \begin{split}
  \epsilon \left[\pd{u}{t} + u \pd{u}{x} + v \pd{u}{y} + w\pd{u}{z}\right] - (1 + \beta \epsilon y)v &= -\pd{\pi}{x} + \Gamma^{2}E\left[\pdd{u}{x} + \pdd{u}{y}\right] + E \pdd{u}{z},\\
  %
  \epsilon \left[\pd{v}{t} + u \pd{v}{x} + v \pd{v}{y} + w\pd{v}{z}\right] + (1 + \beta \epsilon y)u &= -\pd{\pi}{y} + \Gamma^{2}E\left[\pdd{v}{x} + \pdd{v}{y}\right] + E \pdd{v}{z},\\
  %
  \Gamma^{2} \epsilon \left[\pd{w}{t} + u \pd{w}{x} + v \pd{w}{y} + w\pd{w}{z} \right] &= - \pd{\pi}{z} + \Gamma^{4}E\left[\pdd{w}{x} + \pdd{w}{y}\right] + \Gamma^{2}E \pdd{w}{z},\\
  %
  \pd{u}{x} + \pd{v}{y} + \pd{w}{z} &= 0.
 \end{split}
\end{align}
%----------------------------------------------
\subsection{``Outer'' Interior Solution:}
%----------------------------------------------
As suggested in the problem writing $\underline{u} = \underline{u}_{0} + \epsilon \underline{u}_{1} + \hdots$ and $\pi = \pi_{0} + \epsilon \pi_{1} + \hdots$, substituting in Eqn. (\ref{eq:rotating-NS-dimless}), collecting terms at different powers of $\epsilon$, we get:
\begin{align}\label{eq:order-1-outer}
 \begin{split}
 O(1):& \textrm{Geostrophic balance:} \\
 %
 & \pd{\pi_{0}}{x} = v_{0}, \\
 & \pd{\pi_{0}}{y} = -u_{0}, \\
 & \pd{\pi_{0}}{z} = 0,\\
 & \pd{u_{0}}{x} + \pd{v_{0}}{y} + \pd{w_{0}}{z} = 0.\\
 \textrm{BCs}:& u_{0} = v_{0} = w_{0} = 0 \quad \textrm{at} \quad z = -1, \\
 & w_{0} = 0 \quad \textrm{at} \quad z = 0 \\
 & \pd{u_{0}}{z} = \frac{\tau}{\sqrt{E}} \quad \textrm{at} \quad z = 0, \\
 & \pd{v_{0}}{z} = 0 \quad \textrm{at} \quad z = 0, \\
 \end{split}
\end{align}
where $\tau$ is the dimensionless wind-shear at the free surface. The system is not closed, since there is no way of determining $\pi_{0}$ from the current equations. However, we can still draw some conclusions from the leading order geostrophic balance. Substituting the $x$- and $y$- geostrophic balance equations into continuity, we see $-\cancel{\pddmixed{\pi_{0}}{x}{y}} + \cancel{\pddmixed{\pi_{0}}{y}{x}} + \pd{w_{0}}{z} = 0$. This shows that $w_{0}$ is independent of $z$ or $\boxed{w_{0}\equiv w_{0}(x, y, t)}$.\\
%
Taking the $z$-derivative of the $x$- and $y$-geostrophic balance equations and using the fact that $\pd{\pi_{0}}{z} = 0$, we obtain $\pd{u_{0}}{z} = \pd{v_{0}}{z} = 0$. That is, $u_{0}$ and $v_{0}$ are also independent of $z$ or  $\boxed{[u_{0}, v_{0}]\equiv [u_{0}, v_{0}](x, y, t)}$. Now, we go to the next order. Using the distinguished limit, we had related in the two small parameters $\epsilon$ and $E$, such that $ r = \frac{\sqrt{E}}{\epsilon} = O(1)$ as $\epsilon \rightarrow 0$. Keeping this in mind, we write the $O(\epsilon)$ equations as follows:
\begin{align}\label{eq:order-eps-outer}
 \begin{split}
 O(\epsilon):& \\
 %
 & \pd{u_{0}}{t} + u_{0}\pd{u_{0}}{x} + v_{0}\pd{u_{0}}{y} + w_{0}\pd{u_{0}}{z} - v_{1} - \beta y v_{0} = -\pd{\pi_{1}}{x}, \\
 %
 & \pd{v_{0}}{t} + u_{0}\pd{v_{0}}{x} + v_{0}\pd{v_{0}}{y} + w_{0}\pd{v_{0}}{z} + u_{1} + \beta y u_{0} = -\pd{\pi_{1}}{y}, \\
 &%
 \Gamma^{2} \left[ \pd{w_{0}}{t} + u_{0}\pd{w_{0}}{x} + v_{0}\pd{w_{0}}{y} + w_{0}\pd{w_{0}}{z} \right] = -\pd{\pi_{1}}{z}, \\
 %
 &\pd{u_{1}}{x} + \pd{v_{1}}{y} + \pd{w_{1}}{z} = 0.\\
  \textrm{BCs}:& u_{1} = v_{1} = w_{1} = 0 \quad \textrm{at} \quad z = -1, \\
 & w_{1} = 0 \quad \textrm{at} \quad z = 0 \\
 & \pd{u_{1}}{z} = 0 \quad \textrm{at} \quad z = 0, \\
 & \pd{v_{1}}{z} = 0 \quad \textrm{at} \quad z = 0. \\
 \end{split}
\end{align}
%
To eliminate pressure, cross differentiating the $x$- and $y$- moementum equations and subtracting, we obtain:
\begin{align}\label{eq:qg_vorticity}
 \begin{split}
  & \pd{}{x}\left[\pd{v_{0}}{t} + u_{0}\pd{v_{0}}{x} + v_{0}\pd{v_{0}}{y} + w_{0}\pd{v_{0}}{z} + u_{1} + \beta y u_{0} = -\pd{\pi_{1}}{y}\right]\\
  %
  -&\pd{}{y}\left[ \pd{u_{0}}{t} + u_{0}\pd{u_{0}}{x} + v_{0}\pd{u_{0}}{y} + w_{0}\pd{u_{0}}{z} - v_{1} - \beta y v_{0} = -\pd{\pi_{1}}{x}\right]\\
  & --------------------------- \\
  %
  & \pd{\omega_{z0}}{t} + u_{0}\pd{\omega_{z0}}{x} + v_{0}\pd{\omega_{z0}}{y} + w_{0}\pd{\omega_{z0}}{z} + \underbrace{\left( \pd{u_{1}}{x} + \pd{v_{1}}{y}\right)}_{-\pd{w_{1}}{z}} + \beta y \bigg( \cancel{\pd{u_{0}}{x} + \pd{v_{0}}{y}}\bigg) + \beta v_{0} = 0,\\
  %
  & \boxed{\frac{D\omega_{z0}}{Dt} + \beta v_{0} = \pd{\omega_{z0}}{t} + u_{0}\pd{\omega_{z0}}{x} + v_{0}\pd{\omega_{z0}}{y} + w_{0}\pd{\omega_{z0}}{z} + \beta v_{0} = \pd{w_{1}}{z} }.
 \end{split}
\end{align}
Here $\boxed{\omega_{z0} = \pd{v_{0}}{x} - \pd{u_{0}}{y}}$ is the leading-order (\textit{relative}) vertical vorticity. Copying the comment from the assignment here for completion.
``Physically, $\omega_{z0}$ is the leading-order relative z-vorticity: the total z-vorticity equals the sum of $\omega_{z0}$ plus $\epsilon^{-1}$ , where in dimensional terms ``$\epsilon^{-1}$''
corresponds to $f_{0} \equiv 2\Omega \sin{\theta_{0}}$, i.e. the $z$-vorticity fluid particles acquire simply because they are rotating with the Earth. Thus, in terms of vorticity dynamics, the right-hand-side of Eqn.(\ref{eq:qg_vorticity}) is a vortex-stretching term; more specifically, stretching of planetary vortex tubes by the difference in vertical velocities at the ends of these vertically-oriented tubes. (The term involving $\beta$ physically represents the advection of planetary vortex tubes.)''\\
%
To close Eqn.(\ref{eq:qg_vorticity}), we integrate it across the basin. Noting that $u_{0}, v_{0}$(and hence, $\omega_{z0}$) as well as $w_{0}$ are independent of $z$, we get:

%----------------------------------------------
\subsection{``Inner'' Solution at the Upper Free Surface:}
%----------------------------------------------
Redefining boundary layer variables as $\hat{u}, \hat{v}, \hat{w}, \hat{\pi}$ and the co-ordinates as $\hat{x} = x, \hat{y} = y, \hat{z}= z/h(\epsilon)$, where $h(\epsilon) = h^{*}/D$ is the dimensionless thickness of the BL, we obtain:

\begin{align}\label{eq:bl-rotating-NS-dimless}
 \begin{split}
  \epsilon \left[\pd{\hat{u} }{\hat{t} } + \hat{u} \pd{\hat{u} }{\hat{x} } + \hat{v} \pd{\hat{u} }{\hat{y} } + \frac{\hat{w}}{h}\pd{\hat{u}}{\hat{z} }\right] - (1 + \beta \epsilon \hat{y}) \hat{v} &= -\pd{\hat{\pi} }{\hat{x} } + \Gamma^{2}E\left[\pdd{\hat{u}}{\hat{x}} + \pdd{\hat{u}}{\hat{y}}\right] + \frac{E}{h^{2}} \pdd{\hat{u}}{\hat{z} },\\
  %
  \epsilon \left[\pd{\hat{v} }{\hat{t} } + \hat{u} \pd{\hat{v} }{\hat{x} } + \hat{v} \pd{\hat{v} }{\hat{y} } + \frac{\hat{w}}{h}\pd{\hat{v}}{\hat{z} }\right] + (1 + \beta \epsilon \hat{y}) \hat{u} &= -\pd{\hat{\pi} }{\hat{y} } + \Gamma^{2}E\left[\pdd{\hat{v}}{\hat{x}} + \pdd{\hat{v}}{\hat{y}}\right] + \frac{E}{h^{2}} \pdd{\hat{v}}{\hat{z} },\\
  %
  \Gamma^{2} \epsilon \left[\pd{\hat{w} }{\hat{t} } + \hat{u} \pd{\hat{w}}{\hat{x} } + \hat{v} \pd{\hat{w}}{\hat{y}} + \hat{w}\pd{\hat{w}}{\hat{z}} \right] &= - \frac{1}{h}\pd{\hat{\pi}}{\hat{z} } + \Gamma^{4}E\left[\pdd{\hat{w}}{\hat{x}} + \pdd{\hat{w}}{\hat{y}}\right] + \Gamma^{2}\frac{E}{h^{2}} \pdd{\hat{w}}{\hat{z}},\\
  %
  \pd{\hat{u}}{\hat{x}} + \pd{\hat{v}}{\hat{y}} + \frac{1}{h}\pd{\hat{w}}{\hat{z}} &= 0.
 \end{split}
\end{align}
In what follows, since $\hat{x} = x, \hat{y} = y$, we will drop the hats on the $x$ and $y$, but will keep the hat on $\hat{z}$ to remind ourselves that $\hat{z}$ is a BL variable. Before posing an asymptotic expansion and collecting terms, some observations are important. We first choose $h = O(\sqrt{E})$, specifically $h = \sqrt{E} = r\epsilon$, in order to keep the $\hat{z}$-diffusion at the leading order. This is the crucial physics that our interior, outer solution lacks. If we multiply through by $h = \epsilon$ the continuity equation, at leading order, we would obtain: $\boxed{\pd {\hat{w}_{0}}{\hat{z}} = 0}$, i.e., $\hat{w}_{0}$ is independent of $\hat{z}$. Using the upper surface BC, $\hat{w} = 0$ at $\hat{z} = 0$, we get $\hat{w}_{0} = 0$. Hence, the asymptotic sequence for $\boxed{\hat{w} \sim \epsilon \hat{w}_{1} + \hdots}$. \\
%
Now, posing $\hat{u} \sim \hat{u}_{0} + \epsilon \hat{u}_{1} + \hdots$, $\hat{v} \sim \hat{v}_{0} + \epsilon \hat{v}_{1} + \hdots$ and $\hat{\pi} \sim \hat{\pi}_{0} + \epsilon \hat{\pi}_{1} + \hdots$, collecting terms at leading order, we get the following (note: since the leading order in the $\hat{z}$-momentum equation is $O(1/h)$, multiply through by $h = r\epsilon$ and collect terms):

\begin{align}\label{eq:bl-order-1}
 \begin{split}
  & -\hat{v}_{0} = -\pd{\hat{\pi}_{0} }{x} + \pdd{\hat{u}_{0}}{\hat{z} },\\
  %
  &\hat{u}_{0} = -\pd{\hat{\pi}_{0} }{y} + \pdd{\hat{v}_{0}}{\hat{z} },\\
  %
  & 0 = \pd{\hat{\pi}}{\hat{z} }.
 \end{split}
\end{align}
We solve for $\hat{u}_{0}, \hat{v}_{0}$ by obtaining the homogeneous solution (with inhomogeneous BCs: $\frac{1}{\cancel{h} }\pd{\hat{u}_{H}}{\hat{z}} = \frac{\tau}{\cancel{\sqrt{E} }}$) and the inhomogeneous solution (with homogeneous BCs: $\pd{\hat{u}_{I}}{\hat{z}} = 0$). Defining:
\begin{align}
 \begin{split}
  &\hat{u}_{0} = \hat{u}_{H} + \hat{u}_{I}\\
  %
  &\hat{v}_{0} = \hat{v}_{H} + \hat{v}_{I}\\
 \end{split}
\end{align}
%
%----------------------------------------------
\subsubsection{Homogeneous Solution:}
%----------------------------------------------
The homogeneous part of Eqn.(\ref{eq:bl-order-1}) can be written as:
\begin{align}
 \begin{split}
  & -\hat{v}_{H} = \pdd{\hat{u}_{H}}{\hat{z} },\\
  %
  &\hat{u}_{H} = \pdd{\hat{v}_{H}}{\hat{z} }.\\
 \end{split}
\end{align}
%
The two equations can be combined into the form
\begin{equation}\label{eq:bl-complex-hom}
\pdd{\mathcal{U}_{H}}{\hat{z}} - i \mathcal{U}_{H} = 0,
\end{equation}
where $ \mathcal{U}_{H} = \hat{u}_{H} + i \hat{v}_{H}$. The solutions of Eqn.(\ref{eq:bl-complex-hom}) are 
\begin{equation}
 \mathcal{U}_{H} = A e^{\lambda \hat{z}}.
\end{equation}
%
Substituting into Eqn.(\ref{eq:bl-complex-hom}), we get $\lambda^{2} = i$, or $\lambda = \pm \frac{(1+i)}{\sqrt{2}}$. The general solution then becomes:
\begin{equation}\label{eq:bl-complex-hom-gen-soln}
 \mathcal{U}_{H} = A e^{\frac{(1+i)}{\sqrt{2}}\hat{z}} + B e^{-\frac{(1+i)}{\sqrt{2}}\hat{z}}.
\end{equation}
%
When matching with, we'd need to take the outer limit $\hat{z}\rightarrow -\infty$ of this inner solution. Hence, for consistency, we must have $\boxed{B = 0}$, giving 
%
\begin{equation}\label{eq:bl-complex-hom-upper}
 \mathcal{U}_{H} = A e^{\frac{(1+i)}{\sqrt{2}}\hat{z}}.
\end{equation}
As we wrote earlier, we'd impose the inhomogeneous BCs on the homogeneous solution. The BC for $\mathcal{U}_{H}$ at $\hat{z} = 0$ is obtained by combining BCs for $\hat{u}_{H}$ and $\hat{v}_{H}$, namely,
\begin{align}\label{eq:bl-complex-hom-upper-bc}
 \begin{split}
  &\frac{1}{\cancel{h} }\pd{\hat{u}_{H}}{\hat{z}}\bigg|_{\hat{z} = 0} = \frac{\tau}{\cancel{\sqrt{E} }},\\
  %
  & \pd{\hat{u}_{H}}{\hat{z}}\bigg|_{\hat{z} = 0} = \tau, \\
  %
  &\pd{\hat{v}_{H}}{\hat{z}}\bigg|_{\hat{z} = 0} = 0,\\
  %
  &\boxed{\pd{\mathcal{U}_{H}}{\hat{z}}\bigg|_{\hat{z} = 0} = \tau + i(0) }.\\
 \end{split}
\end{align}
This gives
\begin{align}
 \begin{split}
  & A \frac{(1+i)}{\sqrt{2}} = \tau, \\
  %
  & \boxed{A = \tau \frac{(1-i)}{\sqrt{2}} },\\
  %
  &\boxed{\mathcal{U}_{H} = \left[\frac{(\tau-i\tau)}{\sqrt{2}} \right] \exp{\left(\frac{(1+i)}{\sqrt{2}}\hat{z}\right)} }. 
 \end{split}
\end{align}
%
%----------------------------------------------
\subsubsection{Inomogeneous Solution:}
%----------------------------------------------
We can just read-off the inhomogeneous solution. 
\begin{align}\label{eq:bl-complex-inhom-upper}
 \begin{split}
  & \hat{u}_{I} = -\pd{\hat{\pi}_{0} }{y},\\
  %
  & \hat{v}_{I} = \pd{\hat{\pi}_{0} }{x},\\
  %
  &\boxed{\mathcal{U}_{I} = -\pd{\hat{\pi}_{0} }{y} + i \pd{\hat{\pi}_{0} }{x}}. 
 \end{split}
\end{align}
This is consistent since $\pd{\hat{\pi}_{0} }{z} = 0$. Consistency can be checked by substituting the inhomogeneous solution into Eqn.(\ref{eq:bl-order-1}). The inhomogeneous solution satisfies homogeneous BCs $ \pd{\hat{u}_{I}}{\hat{z}}\bigg|_{\hat{z} = 0} =  \pd{\hat{v}_{I}}{\hat{z}}\bigg|_{\hat{z} = 0} = 0$.\\
%
%----------------------------------------------
\subsubsection{Matching: Inner(Outer) $=$ Outer(Inner)}
%----------------------------------------------
\begin{itemize}
 \item Inner limit of the outer geostrophic flow (Eqn. \ref{eq:order-1-outer}):
 \begin{equation}
  \lim_{z \rightarrow 0} u_{0} + i v_{0} = - \pd{\pi_{0}}{y} + i \pd{\pi_{0}}{x}.
 \end{equation}
 %
 \item Outer limit of the inner BL flow:
 \begin{equation}
  \lim_{\hat{z} \rightarrow -\infty} \mathcal{U} = 0 - \pd{\hat{\pi}_{0}}{y} + i \pd{\hat{\pi}_{0}}{x}.
 \end{equation}
 %
\end{itemize}
%
Matching the two limits, we obtain
\begin{align}
 \begin{split}
   & \pd{\pi_{0}}{x} = \pd{\hat{\pi}_{0}}{x},\\
   %
   & \pd{\pi_{0}}{y} = \pd{\hat{\pi}_{0}}{y}. \\
   %
   & \Rightarrow \boxed{\pd{\hat{\pi}_{0}}{x} = v_{0} }\textrm{ and }\\
   %
   & \boxed{ -\pd{\hat{\pi}_{0}}{y} = u_{0}}.   
 \end{split}
\end{align}
%
Substituting these into the BL solution and writing in terms of components, we get
%
\begin{align}\label{eq:bl-full-soln-order-1}
 \hat{u}_{0} \hat{e}_{x} + \hat{v}_{0} \hat{e}_{y} &= \left\{ u_{0} + \frac{\tau}{\sqrt{2}} \exp{\left(\frac{\hat{z}}{\sqrt{2}}\right)} \left[\cos{\left(\frac{\hat{z}}{\sqrt{2}} \right)} + \sin{\left(\frac{\hat{z}}{\sqrt{2}} \right)} \right] \right\}\hat{e}_{x} \\
 %
 &+ \left\{ v_{0} + \frac{\tau}{\sqrt{2}} \exp{\left(\frac{\hat{z}}{\sqrt{2}}\right)} \left[\sin{\left(\frac{\hat{z}}{\sqrt{2}} \right)} - \cos{\left(\frac{\hat{z}}{\sqrt{2}} \right)} \right] \right\}\hat{e}_{y}\\
 %
 &\equiv \underline{u}_{G} + \underline{u}_{E}.
\end{align}
%
where $\underline{u}_{G} = u_{0}\hat{e}_{x} + v_{0}\hat{e}_{y}$ is the interior geostrophic flow and  $\underline{u}_{E} = (\hat{u}_{0}-u_{0})\hat{e}_{x} + (\hat{v}_{0} - v_{0})\hat{e}_{y}$ is the frictionally driven Ekman velocity (Ekman spiral!) confined to the BL. In order to obtain $w_{1}$, we must go to the next order.
%----------------------------------------------
\subsubsection{$w_{1}$ at the Upper Layer:}
%----------------------------------------------
The continuity equation at the next order ($O(1)$ formally, if we multiply through by $\epsilon$, then $O(\epsilon)$) becomes:
\begin{align}
 \begin{split}
  \pd{\hat{w}_{1}}{\hat{z}} &= -\left(\pd{\hat{u}_{0}}{x} + \pd{\hat{v}_{0}}{y} \right)\\
  %
  &= -\left(\cancel{\pd{u_{0}}{x} + \pd{v_{0}}{y} }\right) - \frac{1}{\sqrt{2}}\pd{\tau}{x} \exp{\left(\frac{\hat{z}}{\sqrt{2}}\right)}\left[\cos{\left(\frac{\hat{z}}{\sqrt{2}} \right)} + \sin{\left(\frac{\hat{z}}{\sqrt{2}} \right)} \right]\\
  %
  &-\frac{1}{\sqrt{2}}\pd{\tau}{y} \exp{\left(\frac{\hat{z}}{\sqrt{2}}\right)} \left[\sin{\left(\frac{\hat{z}}{\sqrt{2}} \right)} - \cos{\left(\frac{\hat{z}}{\sqrt{2}} \right)} \right]\\
 \end{split}
\end{align}
%
We need to integrate this from $\hat{z} = 0$ to $\hat{z} = \hat{z}$ to obtain $\hat{w}_{1}\equiv \hat{w}_{1}(\hat{z})$. It is best to go back to complex representation and realize that $\exp{\left(\frac{z}{\sqrt{2}}\right)}\left[\cos{\left(\frac{z}{\sqrt{2}} \right)} + \sin{\left(\frac{z}{\sqrt{2}} \right)} \right] = \mathbf{Re}\left\{\frac{1-i}{\sqrt{2}}\exp{\left(\frac{(1+i)}{\sqrt{2}}z\right)} \right\} $ and $\exp{\left(\frac{z}{\sqrt{2}}\right)} \left[\sin{\left(\frac{z}{\sqrt{2}} \right)} - \cos{\left(\frac{z}{\sqrt{2}} \right)} \right] = \mathbf{Im}\left\{\frac{1-i}{\sqrt{2}}\exp{\left(\frac{(1+i)}{\sqrt{2}}z\right)} \right\}$.

In the notes, we are given: 
\begin{align}
 \begin{split}
   \int_{0}^{\hat{z}} e^{\frac{1+i}{\sqrt{2}} s} ds &= \frac{1}{\sqrt{2}} \left[ -1 + e^{\hat{z}/\sqrt{2}} \left(\cos{\left(\frac{\hat{z}}{\sqrt{2}} \right)} + \sin{\left(\frac{\hat{z}}{\sqrt{2}}\right)} \right) \right]\\
   %
   & + \frac{i}{\sqrt{2}} \left[ 1 - e^{\hat{z}/\sqrt{2}} \left(\cos{\left(\frac{\hat{z}}{\sqrt{2}} \right)} - \sin{\left(\frac{\hat{z}}{\sqrt{2}}\right)} \right) \right], \\
   %
   \left(\frac{1-i}{\sqrt{2}}\right)\int_{0}^{\hat{z}} e^{\frac{1+i}{\sqrt{2}} s} ds &= \left[ e^{\hat{z}/\sqrt{2}} \sin{\left(\frac{\hat{z}}{\sqrt{2}}\right)}  \right]\\
   %
   & + i\left[ 1 - e^{\hat{z}/\sqrt{2}} \cos{\left(\frac{\hat{z}}{\sqrt{2}}\right)}  \right], \\
 \end{split}
\end{align}

\begin{align}
 \begin{split}
  \int_{0}^{\hat{z} }\pd{\hat{w}_{1}}{\hat{s}} d\hat{s} &= - \left(\pd{\tau}{x}\right) \mathbf{Re}\int_{0}^{\hat{z}} \left\{\frac{1-i}{\sqrt{2}}\exp{\left(\frac{(1+i)}{\sqrt{2}}s\right)} \right\} ds \\
  %
  &- \left(\pd{\tau}{y}\right)\mathbf{Im}\int_{0}^{\hat{z}} \left\{\frac{1-i}{\sqrt{2}}\exp{\left(\frac{(1+i)}{\sqrt{2}}s\right)} \right\} ds, \\
  %
  \hat{w}_{1}(\hat{z}) - 0 &= - \left(\pd{\tau}{x}\right)\left[ e^{\hat{z}/\sqrt{2}} \sin{\left(\frac{\hat{z}}{\sqrt{2}}\right)}  \right]\\
  %
  &-\left(\pd{\tau}{y}\right)\left[ 1 - e^{\hat{z}/\sqrt{2}} \cos{\left(\frac{\hat{z}}{\sqrt{2}}\right)}  \right]
 \end{split}
\end{align}

Now, to get the value $\boxed{w_{1}|_{z=0} =\lim_{\hat{z} \rightarrow -\infty}\hat{w}_{1} = - \pd{\tau}{y} = \hat{e}_{z}\cdot(\nabla \times \tau) }$.
%----------------------------------------------
\subsection{``Inner'' Solution at the Bottom Surface:}
%----------------------------------------------
The analysis is exactly same until Eqn.(\ref{eq:bl-complex-hom-gen-soln}). However, in this case, when matching, we will have to take the limit $\hat{z}\rightarrow \infty$, hence $\boxed{A = 0}$. This gives:
%----------------------------------------------
\subsubsection{Homogeneous Solution:}
%----------------------------------------------
\begin{equation}\label{eq:bl-complex-hom-bottom}
 \mathcal{U}_{H} = B e^{-\frac{(1+i)}{\sqrt{2}}z}.
\end{equation}
The boundary conditions are all homogeneous here. 
%------------------------------------------------------
\bibliographystyle{apalike}
%\bibliographystyle{unsrt} % Use for unsorted references  
%\bibliographystyle{plainnat} % use this to have URLs listed in References
%\cleardoublepage
%\bibliography{References/references} % Path to your References.bib file

\bibliography{bib/references} % Path to your References.bib file
 \if@openright\cleardoublepage\else\clearpage\fi
 \cleardoublepage
 \pagestyle{empty}
%--------------------------------------------------------------------
\end{document}
